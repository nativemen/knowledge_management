\documentclass[oneside]{book}

\usepackage{amsmath, amsthm, amssymb, amsfonts}
\usepackage{thmtools}
\usepackage{graphicx}
\usepackage{setspace}
\usepackage{geometry}
\usepackage{float}
\usepackage{hyperref}
\usepackage[utf8]{inputenc}
\usepackage[english]{babel}
\usepackage{framed}
\usepackage[dvipsnames]{xcolor}
\usepackage{environ}
\usepackage{tcolorbox}
\usepackage{todonotes}
\usepackage{booktabs}       % 专业表格线
\usepackage{multirow}       % 合并单元格
\usepackage{caption}        % 表格标题
\usepackage{makecell}
\usepackage{natbib}         % 推荐的参考文献宏包
\tcbuselibrary{theorems,skins,breakable}

\usetikzlibrary{calc,arrows.meta,bending}

\setstretch{1.2}
\geometry{
    textheight=9in,
    textwidth=7in,
    top=1in,
    headheight=12pt,
    headsep=25pt,
    footskip=30pt
}

% Variables
\def\notetitle{Spectral Clustering Algorithms}
\def\noteauthor{
    \textbf{Learner}\\
    {\LaTeX} by Xin Wang\\
}
\def\notedate{Notes}

% The theorem system and user-defined commands
% Theorem System
% The following boxes are provided:
%   Definition:     \defn
%   Theorem:        \thm
%   Lemma:          \lem
%   Corollary:      \cor
%   Proposition:    \prop
%   Claim:          \clm
%   Fact:           \fact
%   Proof:          \pf
%   Example:        \ex
%   Remark:         \rmk (sentence), \rmkb (block)
% Suffix
%   r:              Allow Theorem/Definition to be referenced, e.g. thmr
%   p:              Add a short proof block for Lemma, Corollary, Proposition or Claim, e.g. lemp
%                   For theorems, use \pf for proof blocks

% Definition
\newtcbtheorem[number within=section]{mydefinition}{Definition}
{
    enhanced,
    frame hidden,
    titlerule=0mm,
    toptitle=1mm,
    bottomtitle=1mm,
    fonttitle=\bfseries\large,
    coltitle=black,
    colbacktitle=green!20!white,
    colback=green!10!white,
}{defn}

\NewDocumentCommand{\defn}{m+m}{
    \begin{mydefinition}{#1}{}
        #2
    \end{mydefinition}
}

\NewDocumentCommand{\defnr}{mm+m}{
    \begin{mydefinition}{#1}{#2}
        #3
    \end{mydefinition}
}

% Theorem
\newtcbtheorem[use counter from=mydefinition]{mytheorem}{Theorem}
{
    enhanced,
    frame hidden,
    titlerule=0mm,
    toptitle=1mm,
    bottomtitle=1mm,
    fonttitle=\bfseries\large,
    coltitle=black,
    colbacktitle=cyan!20!white,
    colback=cyan!10!white,
}{thm}

\NewDocumentCommand{\thm}{m+m}{
    \begin{mytheorem}{#1}{}
        #2
    \end{mytheorem}
}

\NewDocumentCommand{\thmr}{mm+m}{
    \begin{mytheorem}{#1}{#2}
        #3
    \end{mytheorem}
}

% Lemma
\newtcbtheorem[use counter from=mydefinition]{mylemma}{Lemma}
{
    enhanced,
    frame hidden,
    titlerule=0mm,
    toptitle=1mm,
    bottomtitle=1mm,
    fonttitle=\bfseries\large,
    coltitle=black,
    colbacktitle=violet!20!white,
    colback=violet!10!white,
}{lem}

\NewDocumentCommand{\lem}{m+m}{
    \begin{mylemma}{#1}{}
        #2
    \end{mylemma}
}

\newenvironment{lempf}{
	{\noindent{\it \textbf{Proof for Lemma}}}
	\tcolorbox[blanker,breakable,left=5mm,parbox=false,
    before upper={\parindent5pt},
    after skip=10pt,
	borderline west={1mm}{0pt}{violet!20!white}]
}{
    \textcolor{violet!20!white}{\hbox{}\nobreak\hfill$\blacksquare$}
    \endtcolorbox
}

\NewDocumentCommand{\lemp}{m+m+m}{
    \begin{mylemma}{#1}{}
        #2
    \end{mylemma}

    \begin{lempf}
        #3
    \end{lempf}
}

% Corollary
\newtcbtheorem[use counter from=mydefinition]{mycorollary}{Corollary}
{
    enhanced,
    frame hidden,
    titlerule=0mm,
    toptitle=1mm,
    bottomtitle=1mm,
    fonttitle=\bfseries\large,
    coltitle=black,
    colbacktitle=orange!20!white,
    colback=orange!10!white,
}{cor}

\NewDocumentCommand{\cor}{+m}{
    \begin{mycorollary}{}{}
        #1
    \end{mycorollary}
}

\newenvironment{corpf}{
	{\noindent{\it \textbf{Proof for Corollary.}}}
	\tcolorbox[blanker,breakable,left=5mm,parbox=false,
    before upper={\parindent15pt},
    after skip=10pt,
	borderline west={1mm}{0pt}{orange!20!white}]
}{
    \textcolor{orange!20!white}{\hbox{}\nobreak\hfill$\blacksquare$}
    \endtcolorbox
}

\NewDocumentCommand{\corp}{m+m+m}{
    \begin{mycorollary}{}{}
        #1
    \end{mycorollary}

    \begin{corpf}
        #2
    \end{corpf}
}

% Proposition
\newtcbtheorem[use counter from=mydefinition]{myproposition}{Proposition}
{
    enhanced,
    frame hidden,
    titlerule=0mm,
    toptitle=1mm,
    bottomtitle=1mm,
    fonttitle=\bfseries\large,
    coltitle=black,
    colbacktitle=yellow!30!white,
    colback=yellow!20!white,
}{prop}

\NewDocumentCommand{\prop}{+m}{
    \begin{myproposition}{}{}
        #1
    \end{myproposition}
}

\newenvironment{proppf}{
	{\noindent{\it \textbf{Proof for Proposition.}}}
	\tcolorbox[blanker,breakable,left=5mm,parbox=false,
    before upper={\parindent15pt},
    after skip=10pt,
	borderline west={1mm}{0pt}{yellow!30!white}]
}{
    \textcolor{yellow!30!white}{\hbox{}\nobreak\hfill$\blacksquare$}
    \endtcolorbox
}

\NewDocumentCommand{\propp}{+m+m}{
    \begin{myproposition}{}{}
        #1
    \end{myproposition}

    \begin{proppf}
        #2
    \end{proppf}
}

% Claim
\newtcbtheorem[use counter from=mydefinition]{myclaim}{Claim}
{
    enhanced,
    frame hidden,
    titlerule=0mm,
    toptitle=1mm,
    bottomtitle=1mm,
    fonttitle=\bfseries\large,
    coltitle=black,
    colbacktitle=pink!30!white,
    colback=pink!20!white,
}{clm}


\NewDocumentCommand{\clm}{m+m}{
    \begin{myclaim*}{#1}{}
        #2
    \end{myclaim*}
}

\newenvironment{clmpf}{
	{\noindent{\it \textbf{Proof for Claim.}}}
	\tcolorbox[blanker,breakable,left=5mm,parbox=false,
    before upper={\parindent15pt},
    after skip=10pt,
	borderline west={1mm}{0pt}{pink!30!white}]
}{
    \textcolor{pink!30!white}{\hbox{}\nobreak\hfill$\blacksquare$}
    \endtcolorbox
}

\NewDocumentCommand{\clmp}{m+m+m}{
    \begin{myclaim*}{#1}{}
        #2
    \end{myclaim*}

    \begin{clmpf}
        #3
    \end{clmpf}
}

% Fact
\newtcbtheorem[use counter from=mydefinition]{myfact}{Fact}
{
    enhanced,
    frame hidden,
    titlerule=0mm,
    toptitle=1mm,
    bottomtitle=1mm,
    fonttitle=\bfseries\large,
    coltitle=black,
    colbacktitle=purple!20!white,
    colback=purple!10!white,
}{fact}

\NewDocumentCommand{\fact}{+m}{
    \begin{myfact}{}{}
        #1
    \end{myfact}
}


% Proof
\NewDocumentCommand{\pf}{+m}{
    \begin{proof}
        [\noindent\textbf{Proof.}]
        #1
    \end{proof}
}

% Example
\newenvironment{example}{%
    \par
    \vspace{5pt}
	\begin{minipage}{0.9\textwidth}
		\noindent\textbf{Example.}
		\tcolorbox[blanker,breakable,left=5mm,parbox=false,
	    before upper={\parindent15pt},
	    after skip=10pt,
		borderline west={1mm}{0pt}{cyan!10!white}]
}{%
		\endtcolorbox
	\end{minipage}
    \vspace{5pt}
}

\NewDocumentCommand{\ex}{+m}{
    \begin{example}
        #1
    \end{example}
}


% Remark
\NewDocumentCommand{\rmk}{+m}{
    {\it \color{blue!50!white}#1}
}

\newenvironment{remark}{
    \par
    \vspace{5pt}
    \begin{minipage}{0.9\textwidth}
        {\par\noindent{\textbf{Remark.}}}
        \tcolorbox[blanker,breakable,left=5mm,
        before skip=10pt,after skip=10pt,
        borderline west={1mm}{0pt}{cyan!10!white}]
}{
        \endtcolorbox
    \end{minipage}
    \vspace{5pt}
}

\NewDocumentCommand{\rmkb}{+m}{
    \begin{remark}
        #1
    \end{remark}
}

\newcommand{\lcm}{\operatorname{lcm}}



% ------------------------------------------------------------------------------

\begin{document}
\title{
    \textbf{
        \LARGE{\notetitle} \vspace*{10\baselineskip}
    }
}
\author{\noteauthor}
\date{\notedate}

\maketitle
\newpage

\tableofcontents  % 目录
\listoffigures    % 图目录
\listoftables     % 表目录
\newpage

% ------------------------------------------------------------------------------

\chapter{Graph theory}

\section{Basic terminologies}{
    \defn{Network or Graph}{
        A network (or graph) consists of a set of nodes (or vertices, actors) and a set of edges (or links, ties) that connect those nodes.
    }

    \defn{Neighbor}{
        Node j is called a neighbor of node i if (and only if) node i is connected to node j.
    }

    \defn{Degree}{
        The number of edges connected to a node. Node i’s degree is often written as $\deg(i)$.
    }

    \defn{Walk}{
        A list of edges that are sequentially connected to form a continuous route on a network.
    }

    \defn{Trail}{
        A walk that doesn’t go through any edge more than once.
    }

    \defn{Path}{
        A walk that doesn’t go through any node (and therefore any edge, too) more than once.
    }

    \defn{Cycle}{
        A walk that starts and ends at the same node without going through any node more than once on its way.
    }

    \defn{Subgraph}{
        Part of the graph.
    }

    \defn{Connected graph}{
        A graph in which a path exists between any pair of nodes.
    }

    \defn{Connected component}{
        A subgraph of a graph that is connected within itself but not connected to the rest of the graph.
    }

\section{Graphs}{
    \defn{Complete graph}{
        A graph in which any pair of nodes are connected.
    }

    \defn{Regular graph}{
        A graph in which all nodes have the same degree.Every complete graph is regular.
    }

    \defn{Bipartite (n-partite) graph}{
        A graph whose nodes can be divided into two (or n) groups so that no edge connects nodes within each group.
    }

    \defn{Tree graph}{
        A graph in which there is no cycle. A graph made of multiple trees is called a forest graph. Every tree or forest graph is bipartite.
    }

    \defn{Planar graph}{
        A graph that can be graphically drawn in a two-dimensional plane with no edge crossings. Every tree or forest graph is planar.

    }
}

\section{Edges}{
    \defn{Undirected edge}{
        A symmetric connection between nodes. If node i is connected to node j by an undirected edge, then node j also recognizes node i as its neighbor. A graph made of undirected edges is called an undirected graph. The adjacency matrix of an undirected graph is always symmetric.
    }

    \defn{Directed edge}{
        An asymmetric connection from one node to another. Even if node i is connected to node j by a directed edge, the connection isn’t necessarily reciprocated from node j to node i. A graph made of directed edges is called a directed graph. The adjacency matrix of a directed graph is generally asymmetric.
    }

    \defn{Unweighted edge}{
        An edge without any weight value associated to it. There are only two possibilities between a pair of nodes in a network with unweighted edges; whether there is an edge between them or not. The adjacency matrix of such a network is made of only 0’s and 1’s.
    }

    \defn{Weighted edge}{
        An edge with a weight value associated to it. A weight is usually given by a non-negative real number, which may represent a connection strength or distance between nodes, depending on the nature of the system being modeled. The definition of the adjacency matrix can be extended to contain those edge weight values for networks with weighted edges. The sum of the weights of edges connected to a node is often called the node strength, which corresponds to a node degree for unweighted graphs.
    }

    \defn{Multiple edges}{
        Edges that share the same origin and destination. Such multiple edges connect two nodes more than once.
    }

    \defn{Self-loop}{
        An edge that originates and ends at the same node.
    }

    \defn{Simple graph}{
        A graph that doesn’t contain directed, weighted, or multiple edges, or self-loops. Traditional graph theory mostly focuses on simple graphs.
    }

    \defn{Multigraph}{
        A graph that may contain multiple edges. Many mathematicians also allow multigraphs to contain self-loops. Multigraphs can be undirected or directed.
    }

    \defn{Hyperedge}{
        A generalized concept of an edge that can connect any number of nodes at once, not just two. A graph made of hyperedges is called a hypergraph (not covered in this textbook).
    }
}

\section{Matrices}{
    \defn{Degree matrix}{
        Given a graph \(G=(V,E)\) with $|V|=n$, $V$ is the vertex set, and $E$ is the edge set, the degree matrix $D$ for $G$ is a $n \times n$ diagonal matrix defined as
        \[
            D_{ij}:=
                \left\{
                    \begin{array}{ll}
                        \deg(v_{i}) & i=j \\
                        0           & i \neq j
                    \end{array}
                \right.
        \]
        where the degree $\deg(v_{i})$ of a vertex counts the number of times an edge terminates at that vertex.\\
        In an \textbf{undirected graph}, this means that each loop increases the degree of a vertex by two.\\
        In a \textbf{directed graph}, the term degree may refer either to indegree (the number of incoming edges at each vertex) or outdegree (the number of outgoing edges at each vertex)
    }

    \defn{Adjacency matrix}{
        Given a graph $G=(V,E)$ with $|V|=n$, $V$ is the vertex set, and $E$ is the edge set, the adjacency matrix $A$ for $G$ is a $n \times n$ matrix defined as
        \[
            A_{ij}:=
                \left\{
                    \begin{array}{ll}
                        1 & (v_{i},v_{j}) \in E \\
                        0 & (v_{i},v_{j}) \notin E
                    \end{array}
                \right.
        \]
        where $(v_i,v_j)$ is an edge in $E$.\\
        A matrix with rows and columns labeled by nodes, whose i-th row, j-th column component  $a_{ij}$ is 1 if node i is a neighbor of node j, or 0 otherwise.
    }

    \cor{
        The diagonal elements of the matrix are all zero, since edges from a vertex to itself (loops) are not allowed in simple graphs.
    }

    \cor{
        The adjacency matrix is a square matrix with dimensions equal to the number of vertices in the graph.
    }

    \cor{
        For an undirected graph, the matrix is symmetric, meaning.
        \[A_{ij}=A_{ji}\]
        For a directed graph, the matrix may not be symmetric.
    }

    \cor{
        The degree matrix $D$ and the adjacency matrix $A$ of a graph $G$ are related by the following equation:
        \[D=A+A^T\]
        where $A^T$ is the transpose of $A$.
        This equation can be derived by considering the degree of each vertex in the graph.
        The degree of a vertex is the sum of the weights of all edges incident to that vertex.
        In the adjacency matrix, each row or column corresponds to a vertex, and the entries in that row or column represent the weights of the edges incident to that vertex.
        Therefore, the sum of the entries in a row or column of the adjacency matrix is equal to the degree of the corresponding vertex.
        Since the degree matrix is a diagonal matrix with the degrees of the vertices on the diagonal, the sum of the entries in each row or column of the adjacency matrix is equal to the corresponding diagonal entry in the degree matrix.
        Therefore, the adjacency matrix and the degree
        matrix are related by the equation $D=A+A^T$.
    }

    \defn{Adjacency list}{
        Adjacency list A list of lists of nodes whose i-th component is the list of node i’s neighbors.
    }

    \defn{Laplacian matrix}{
        Given a simple $G$ with $n$ vertices $v_{1},\cdots,v_{n}$, its Laplacian matrix $L{n\times n}$ is defined element-wise as
        \[
            L_{ij}:=
                \left\{
                    \begin{array}{ll}
                        \deg(v_i) & i=j \\
                        -1        & i \neq j \text{ and } (v_{i},v_{j}) \in E \\
                        0         & \text{otherwise}
                    \end{array}
                \right.
        \]
        or equivalently by the matrix
        \[L=D-A\]
        where $D$ is the degree matrix and $A$ is the adjacency matrix of the graph.\\
        The Laplacian matrix is a symmetric matrix, and its eigenvalues are real and non-negative.
    }

    \cor{
        For the undirected graph that both the adjacency matrix and the Laplacian matrix are symmetric, and that row- and column-sums of the Laplacian matrix are all zeros (which directly implies that the Laplacian matrix is singular).
    }
}

\chapter{Classical Spectral Clustering Algorithm}

\section{Classical Spectral Clustering Algorithm}{
}

% \bibliographystyle{plain}  % 参考文献样式,可选:plain, unsrt, alpha, abbrv
% \bibliography{spectral_clustering_algorithms}  % 引用 references.bib 文件

\end{document}
