\documentclass[oneside]{book}

\usepackage{amsmath, amsthm, amssymb, amsfonts}
\usepackage{thmtools}
\usepackage{graphicx}
\usepackage{setspace}
\usepackage{geometry}
\usepackage{float}
\usepackage{hyperref}
\usepackage[utf8]{inputenc}
\usepackage[english]{babel}
\usepackage{framed}
\usepackage[dvipsnames]{xcolor}
\usepackage{environ}
\usepackage{tcolorbox}
\usepackage{todonotes}
\usepackage{booktabs}       % 专业表格线
\usepackage{multirow}       % 合并单元格
\usepackage{caption}        % 表格标题
\usepackage{makecell}
\usepackage{natbib}         % 推荐的参考文献宏包
\tcbuselibrary{theorems,skins,breakable}

\usetikzlibrary{calc,arrows.meta,bending}

\setstretch{1.2}
\geometry{
    textheight=9in,
    textwidth=7in,
    top=1in,
    headheight=12pt,
    headsep=25pt,
    footskip=30pt
}

% Variables
\def\notetitle{Spectral Clustering Algorithms}
\def\noteauthor{
    \textbf{Learner}\\
    {\LaTeX} by Xin Wang\\
}
\def\notedate{Notes}

% The theorem system and user-defined commands
\input{theorems.tex}
\input{commands.tex}

% ------------------------------------------------------------------------------

\begin{document}
\title{
    \textbf{
        \LARGE{\notetitle} \vspace*{10\baselineskip}
    }
}
\author{\noteauthor}
\date{\notedate}

\maketitle
\newpage

\tableofcontents  % 目录
\listoffigures    % 图目录
\listoftables     % 表目录
\newpage

% ------------------------------------------------------------------------------

\chapter{Basic Concepts}

\section{Graph theory}{
    \defn{Degree matrix}{
        Given a graph $G=(V,E)$ with $|V|=n$, $V$ is the vertex set, and $E$ is the edge set, the degree matrix $D$ for $G$ is a $n \times n$ diagonal matrix defined as
        \[
            D_{ij}:=
                \left\{
                    \begin{array}{ll}
                        \deg(v_i) & i=j \\
                        0 & i \neq j
                    \end{array}
                \right.
        \]
        where the degree $\deg(v_{i})$ of a vertex counts the number of times an edge terminates at that vertex.\\
        In an \textbf{undirected graph}, this means that each loop increases the degree of a vertex by two.\\
        In a \textbf{directed graph}, the term degree may refer either to indegree (the number of incoming edges at each vertex) or outdegree (the number of outgoing edges at each vertex)
    }

    \defn{Adjacency matrix}{
        Given a graph $G=(V,E)$ with $|V|=n$, $V$ is the vertex set, and $E$ is the edge set, the adjacency matrix $A$ for $G$ is a $n \times n$ matrix defined as
        \[
        A_{ij}:=
        \left\{
            \begin{array}{ll}
                1 & (v_i,v_j) \in E \\
                0 & (v_i,v_j) \notin E
            \end{array}
        \right.
        \]
        where $(v_i,v_j)$ is an edge in $E$.
    }

    \defn{Laplacian matrix}{

    }
}

\chapter{Classical Spectral Clustering Algorithm}

\section{Classical Spectral Clustering Algorithm}{
}

% \bibliographystyle{plain}  % 参考文献样式,可选:plain, unsrt, alpha, abbrv
% \bibliography{spectral_clustering_algorithms}  % 引用 references.bib 文件

\end{document}
