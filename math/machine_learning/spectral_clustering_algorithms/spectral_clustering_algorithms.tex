\documentclass[oneside]{book}

\usepackage{amsmath, amsthm, amssymb, amsfonts}
\usepackage{thmtools}
\usepackage{graphicx}
\usepackage{setspace}
\usepackage{geometry}
\usepackage{float}
\usepackage{hyperref}
\usepackage[utf8]{inputenc}
\usepackage[english]{babel}
\usepackage{framed}
\usepackage[dvipsnames]{xcolor}
\usepackage{environ}
\usepackage{tcolorbox}
\usepackage{todonotes}
\usepackage{booktabs}       % 专业表格线
\usepackage{multirow}       % 合并单元格
\usepackage{caption}        % 表格标题
\usepackage{makecell}
\usepackage{natbib}         % 推荐的参考文献宏包
\tcbuselibrary{theorems,skins,breakable}

\usetikzlibrary{calc,arrows.meta,bending}

\setstretch{1.2}
\geometry{
    textheight=9in,
    textwidth=7in,
    top=1in,
    headheight=12pt,
    headsep=25pt,
    footskip=30pt
}

% Variables
\def\notetitle{Spectral Clustering Algorithms}
\def\noteauthor{
    \textbf{Learner}\\
    {\LaTeX} by Xin Wang\\
    }
\def\notedate{Notes}

% The theorem system and user-defined commands
\input{theorems.tex}
\input{commands.tex}

% ------------------------------------------------------------------------------

\begin{document}
\title{\textbf{
    \LARGE{\notetitle} \vspace*{10\baselineskip}}
    }
\author{\noteauthor}
\date{\notedate}

\maketitle
\newpage

\tableofcontents
\newpage

% ------------------------------------------------------------------------------

\chapter{Basic Concepts}

\section{Graph theory}{
    \defn{Degree matrix}{

    }
    \defn{Laplacian matrix}{

    }
}

\chapter{Classical Spectral Clustering Algorithm}

\section{Classical Spectral Clustering Algorithm}{
}

% \bibliographystyle{plain}  % 参考文献样式,可选:plain, unsrt, alpha, abbrv
% \bibliography{spectral_clustering_algorithms}  % 引用 references.bib 文件

\end{document}
